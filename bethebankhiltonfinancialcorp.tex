% Options for packages loaded elsewhere
\PassOptionsToPackage{unicode}{hyperref}
\PassOptionsToPackage{hyphens}{url}
%
\documentclass[
]{book}
\usepackage{lmodern}
\usepackage{amssymb,amsmath}
\usepackage{ifxetex,ifluatex}
\ifnum 0\ifxetex 1\fi\ifluatex 1\fi=0 % if pdftex
  \usepackage[T1]{fontenc}
  \usepackage[utf8]{inputenc}
  \usepackage{textcomp} % provide euro and other symbols
\else % if luatex or xetex
  \usepackage{unicode-math}
  \defaultfontfeatures{Scale=MatchLowercase}
  \defaultfontfeatures[\rmfamily]{Ligatures=TeX,Scale=1}
\fi
% Use upquote if available, for straight quotes in verbatim environments
\IfFileExists{upquote.sty}{\usepackage{upquote}}{}
\IfFileExists{microtype.sty}{% use microtype if available
  \usepackage[]{microtype}
  \UseMicrotypeSet[protrusion]{basicmath} % disable protrusion for tt fonts
}{}
\usepackage{xcolor}
\IfFileExists{xurl.sty}{\usepackage{xurl}}{} % add URL line breaks if available
\IfFileExists{bookmark.sty}{\usepackage{bookmark}}{\usepackage{hyperref}}
\hypersetup{
  pdftitle={Be The Bank},
  pdfauthor={Jack W Hilton},
  hidelinks,
  pdfcreator={LaTeX via pandoc}}
\urlstyle{same} % disable monospaced font for URLs
\usepackage{longtable,booktabs}
% Correct order of tables after \paragraph or \subparagraph
\usepackage{etoolbox}
\makeatletter
\patchcmd\longtable{\par}{\if@noskipsec\mbox{}\fi\par}{}{}
\makeatother
% Allow footnotes in longtable head/foot
\IfFileExists{footnotehyper.sty}{\usepackage{footnotehyper}}{\usepackage{footnote}}
\makesavenoteenv{longtable}
\usepackage{graphicx,grffile}
\makeatletter
\def\maxwidth{\ifdim\Gin@nat@width>\linewidth\linewidth\else\Gin@nat@width\fi}
\def\maxheight{\ifdim\Gin@nat@height>\textheight\textheight\else\Gin@nat@height\fi}
\makeatother
% Scale images if necessary, so that they will not overflow the page
% margins by default, and it is still possible to overwrite the defaults
% using explicit options in \includegraphics[width, height, ...]{}
\setkeys{Gin}{width=\maxwidth,height=\maxheight,keepaspectratio}
% Set default figure placement to htbp
\makeatletter
\def\fps@figure{htbp}
\makeatother
\setlength{\emergencystretch}{3em} % prevent overfull lines
\providecommand{\tightlist}{%
  \setlength{\itemsep}{0pt}\setlength{\parskip}{0pt}}
\setcounter{secnumdepth}{5}
\usepackage{wrapfig}
\usepackage{indentfirst}

\title{Be The Bank}
\author{Jack W Hilton}
\date{2021-08-27}

\begin{document}
\maketitle

{
\setcounter{tocdepth}{1}
\tableofcontents
}
\hypertarget{disclaimer}{%
\chapter*{Disclaimer}\label{disclaimer}}


The information in this book is not intended to serve as legal or professional advice.
It is provided with the understanding that the author and the publisher are not engaged in rendering legal, accounting, or other professional service.

The reader should not rely on the material in this text.
If legal, professional, or expert assistance is required, the services of a competent professional should be sought.

© Copyright 2020 by Jack Hilton

All rights reserved.

\begin{center}\rule{0.5\linewidth}{0.5pt}\end{center}

~~Jack Hilton, President\\
\hspace*{0.333em}\hspace*{0.333em}Hilton Financial Corporation\\
\hspace*{0.333em}\hspace*{0.333em}11024 N 28th Drive, Suite 170\\
\hspace*{0.333em}\hspace*{0.333em}Phoenix, AZ 85029-4329\\
\hspace*{0.333em}\hspace*{0.333em}\href{https://www.hiltonloans.com/}{www.hiltonloans.com}

\begin{center}\rule{0.5\linewidth}{0.5pt}\end{center}

\hypertarget{acknowledgements}{%
\chapter*{Acknowledgements}\label{acknowledgements}}


For this book the following references works were used:
The Chicago Manual of Style (CMS; 17th edition),
Words Into Type (3rd edition),
and Merriam-Webster's Collegiate Dictionary (11th edition, online).

This book has been in the making for several years.
I wish to thank all of those who have helped in anyway in editing, comments, and contributions to the text of this book.

Stephen Haggard reviewed an early manuscript of this work and made several valuable comments.
Steve was with me when I was a director and chairman of Rocky Mountain Bank.
We were grooming him to be the next president.
Banks always need a succession plan, and Steve was ours.
He is now the president, CEO, and director of Metro Phoenix Bank.
A more dedicated, hardworking, and honest person is hard to find in the banking world.
I am glad to have been working with Steve for so many years and serving with him as a director and at Metro Phoenix Bank.

My family has been very helpful in many ways.
My children grew up in the real estate and lending world
and many of them have contributed to reading and editing past versions of the manuscript.

My son Nicholas Hilton was with Hilton Financial Corporation for many years.
At the beginning of the Great Recession,
he decided he wanted to move to a cooler climate,
so off to Denver he and his family went.
He is currently an underwriter for a mortgage company and also holds a real estate license in Arizona.
Nick helped with organizing and editing the early manuscripts of this work, for which I am grateful.

My son, Jack Hilton II, has helped compile and edit this book,
bringing it from the early manuscripts to a finished product.
Jack added a lot of content to this book from his personal knowledge and research.
Jack is a loan officer at Hilton Financial Corporation
and has been with the company for many years.
He knows the business inside and out,
and I am very grateful for the dedication he has shown on this project.

My son Beau Hilton and daughter, Charli Hilton Hayes,
have both reviewed and made comments on earlier manuscripts.
Beau is a doctor, currently serving his residency and internship at Vanderbilt,
and Charli is currently working at Metro Phoenix Bank.
Both are high performers in their fields.

My wife, Marci, has been an inspiration to this work from the beginning.
When we were first married, I had to learn to not talk real estate and finance in every conversation.
There are a lot more romantic conversations to be spoken than talking shop. M
arci can hold her own with any conversation in the subject of real estate or finance,
and it is fun to watch and listen.
We have five wonderful children.
We had three children and then an eight-year gap, and then two more.
Marci likes to tell people she was a single mom with the first three children.
It took me that long to figure things out and get my head above the water of learning,
even though we are never really done learning.
Heaven forbid.

\begin{center}\rule{0.5\linewidth}{0.5pt}\end{center}

\hypertarget{authormessage}{%
\chapter*{A Message from the Author}\label{authormessage}}


Lending money to others is one of the oldest businesses in the world.
You read about it in the Bible and many other ancient books.
In this book, we'll discuss a wide array of investments associated with moneylending and banking.
We'll briefly discuss the history of banking.
We'll talk about the types of banks extant today.
We'll talk about starting your own bank,
whether that will be a large mortgage banking operation,
or a small, self-directed lending business.
I'll give you an overview of the common concerns bankers and private lenders deal with daily.
We'll review common ways to protect your investments and how the current laws either enable or limit your investments.

This publication is designed to provide general information about the subject matter covered.
Bearing in mind that laws and practices vary from city to city, state to state, and country to country,
and also because each individual situation will be different,
the reader should consult with appropriate advisors regarding any specific situation.

The author has taken reasonable precautions in the preparation of this book
and believes the material is accurate as of the date it was written.
However, neither the author nor the publisher assume any responsibility for any error or omission.
Nor can we warrant accuracy against the progress of time.
Legislation and industry practices may further change the face of lending with each day that passes.
The author and publisher specifically disclaim any liability resulting from the use or application of the information contained in this book.
The information in this book is not intended to serve as legal or professional advice related to any individual situation.

With the above legal disclaimer out of the way,
we're free to jump into the fascinating world of moneylending.
But first, I'll tell you about how I got my start in banking
and some of the important milestones and anecdotes from my career.

\begin{center}\rule{0.5\linewidth}{0.5pt}\end{center}

\hypertarget{chapter-template}{%
\chapter*{CHAPTER TEMPLATE}\label{chapter-template}}


Putting the \texttt{\{-\}} after the section name makes it so the chapter title will not include a number, if you like that look better.

\texttt{-\/-\/-} adds a horizontal line, as below, if you'd like to have a separator between sections.

\begin{center}\rule{0.5\linewidth}{0.5pt}\end{center}

\hypertarget{subsection-title}{%
\section{SUBSECTION TITLE}\label{subsection-title}}

The more \texttt{\#\#\#} you add, the farther down you go into subsections.
See how this one got a number, because we didn't add the \texttt{\{-\}}?

\hypertarget{picturetemplate}{%
\section*{Addng pictures}\label{picturetemplate}}


The following code will insert a picture.

\begin{figure}
\centering
\includegraphics{images/macarons.jpg}
\caption{\label{fig:example-macarons}A pretty picture of tasty food}
\end{figure}

The phrase after the first \texttt{r} must be hypenated if it's more than one word (spaces will confuse it),
it's a reference you can use later if you want to point the reader to this specific image.

\texttt{fig.cap} allows you to add a caption.

\texttt{echo\ =\ FALSE} makes it so the code that inserts the picture does not show up in the final product (switch it to \texttt{TRUE} or leave it out to see what happens).

the \texttt{knitr}\ldots{} section contains a path to the image.
It's probably easiest to upload all the images in the \texttt{images} folder,
and give them all short and easy names.

(If you're looking at the source code, notice that we added \texttt{\#picturetemplate} after the \texttt{\{-} part of the section title.
This allows us to reference this section elsewhere in the book, with a clickable link, if wanted.
If we end up changing the order of things, it will automatically detect that and correct the link and numbering for it,
so it's better than using a page number or something like that.)

\hypertarget{subsubsection-title}{%
\subsection*{SUBSUBSECTION TITLE}\label{subsubsection-title}}


Such nesting!

\begin{center}\rule{0.5\linewidth}{0.5pt}\end{center}

\hypertarget{notes}{%
\chapter*{NOTES}\label{notes}}


Notes could go here.

\begin{center}\rule{0.5\linewidth}{0.5pt}\end{center}

\hypertarget{chapter-template-1}{%
\chapter*{CHAPTER TEMPLATE}\label{chapter-template-1}}


Here's another chapter file.

You can split each chapter into files, or use one big file. Whatever you'd like!

When you add a chapter, go to the \texttt{\_bookdown.yml} file and add it to the list of files to be included in the book, in the order you'd like them.

E.g.

\texttt{rmd\_files:\ {[}"index.Rmd",\ "01-chapter.Rmd,\ "02-chapter.Rmd"{]}}

This lets you rearrange the book easily, by rearranging the order of items in the list:

\texttt{rmd\_files:\ {[}"index.Rmd",\ "02-chapter.Rmd,\ "01-chapter.Rmd"{]}}

The file names don't really matter,
since you're specifying the order of files manually,
but most folks use the \texttt{xx-name} style.

It might also make sense to not use numbers, and instead titles.

\texttt{rmd\_files:\ {[}"index.Rmd",\ "my-start.Rmd,\ "history-banking.Rmd"{]}}

\begin{center}\rule{0.5\linewidth}{0.5pt}\end{center}

\hypertarget{subsection-title-1}{%
\section{SUBSECTION TITLE}\label{subsection-title-1}}

The more \texttt{\#\#\#} you add, the farther down you go into subsections.
See how this one got a number, because we didn't add the \texttt{\{-\}}?

\hypertarget{addng-pictures}{%
\section*{Addng pictures}\label{addng-pictures}}


The following code will insert a picture.

\begin{figure}
\centering
\includegraphics{images/coins.jpg}
\caption{\label{fig:example-coins}Durable!}
\end{figure}

\hypertarget{subsubsection-title-1}{%
\subsection*{SUBSUBSECTION TITLE}\label{subsubsection-title-1}}


Such nesting!

Look, a \protect\hyperlink{picturetemplate}{clickable link} to somewhere else in the book!

\begin{center}\rule{0.5\linewidth}{0.5pt}\end{center}

\hypertarget{notes-1}{%
\chapter*{NOTES}\label{notes-1}}


Notes could go here.

\begin{center}\rule{0.5\linewidth}{0.5pt}\end{center}

\hypertarget{bio}{%
\chapter*{Biography}\label{bio}}


Jack W. Hilton has been facilitating the financing, acquisition, and development of real estate projects
in the Arizona market since 1976.
His credentials include 45 years of experience in the local real estate market,
where he has witnessed and participated in Arizona's dramatic growth.

Mr.~Hilton has been active as an owner of mortgage companies,
a real estate brokerage and management company,
a licensed broker dealer company,
as well as construction and development companies.
His infatuation with real estate industry started
when he purchased his first small apartment complex at the age of 19.
Shortly thereafter, he built his first spec home,
which led to the construction and management of numerous small apartment buildings during the early to mid-1980s.

Mr.~Hilton became a real estate lender in the early 1980s to assist, in part,
the planning and acquiring, and financing for his own projects.
As family members began to develop their own real estate projects and needed access to financing,
Mr.~Hilton decided to obtain his Mortgage Brokers License.
He received his first Mortgage Brokers License on October 1, 1983,
which led to bigger and greater involvement in this exciting field.

Some of Mr.~Hilton's accomplishments include:

\begin{itemize}
\tightlist
\item
  Funding more than \$1 billion of private money loans since 1990.
\item
  Being instrumental in turning a local bank under regulatory enforcement in 1996 into a profitable financial institution. Rocky Mountain Bank went from a single branch location to seven, including five branches that were located inside Super Walmart locations. Under his leadership the bank grew from \$20 million in assets to \$130 million during an eight-year period. Mr.~Hilton was a director and chairman of the bank. He transformed the bank from a privately owned institution into a public bank listed on Nasdaq's OTC Pink bulletin board.
\item
  Being instrumental in formation of Metro Phoenix Bank in 2006 where he currently serves as the chairman and a director of the Bank.
\item
  Serving previously as the president, director, treasurer, and secretary of numerous HOA associations.
\item
  Previously owning and operating a licensed broker dealer company specializing in private equity offerings.
\item
  Raising millions of dollars in the syndication of mortgage loans and real estate projects.
\item
  Developing numerous subdivisions through various stages of planning---from the initial design and preliminary platting process, to acquiring subdivisions that already had preliminary or final plats---and then completing the actual onsite development.
\end{itemize}

Mr.~Hilton is currently the chairman and president of Hilton Financial Corporation,
acting as the designated broker for Hilton Realty Corporation, and chairman and president of Phoenix MGP, Inc.,
which serves as the general partner and/or manager of numerous entities.
He is a past director and president of the Arizona Private Lenders Association.

To contact Jack Hilton with questions or comments, you can email him at \href{mailto:jack@hiltoncorp.com}{\nolinkurl{jack@hiltoncorp.com}}.

\end{document}
